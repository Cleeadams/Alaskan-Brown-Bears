% There are many arguments for why the global temperature of the earth is changing so dramatically. After looking at recorded data, scientists were able to construct models to reveal what the temperatures were like a 1000 years ago \cite{crowley2000causes}. 
% They also found that after the end of the late 19$^{th}$ century the temperature of the earth started to increase \cite{hansen2006global}. Scientists have discovered the reason for the recent spike in temperature is because of the increase in greenhouse gases \cite{crowley2000causes}.
% Which are gases that keep heat from escaping the atmosphere such as carbon dioxide, methane, nitrous oxide, and flourinated gases \cite{epagreen}. 
% It is agreed upon now by most climatologists that the cause of the this sudden increase of greenhouse gases is due to humans \cite{cook2016consensus}. 
% Flourinated gases are man made gases from industrial processes and are quite powerful in keeping the heat from escaping. 
% Fortunately, they are produced in small quantities, but the emissions of these gases have been on the rise since 1990, so they are not something to be overlooked \cite{epagreen}.
% In 2019, the EPA reported that carbon dioxide made up 80\% of the greenhouse gas emissions \cite{epagreen}. 
% Even though flourinated gases are strictly man made, carbon dioxide is the leading human emitted greenhouse gas. 
% Because humans depend on burning fossil fuels for energy such as transportation, electricity, industrial processes, and residential/commercial uses, carbon dioxide will continue to be the main source of greenhouse gas emissions \cite{epagreen}.
% Trees, plants, and the ocean are part of the carbon cycle which help to remove the carbon dioxide in our atmosphere.
% Since humans are tearing down forests and polluting the ocean, they are also disrupting the filtration system the earth uses to remove carbon dioxide from our atmosphere \cite{epagreen}.
% Carbon dioxide emissions have been in a slow decline over the past 15 years, so there is a possibility to see the change in temperature slow down in the near future \cite{epagreen}.
