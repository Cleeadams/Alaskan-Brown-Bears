In this chapter, we proposed a salmon growth rate function that depends on water temperature.
We used river temperature data from the United States Geological Survey (USGS) to design a function that models the increase in Alaskan river temperature over time, which can be seen below,
\begin{equation*}
    T(t) = a*t + b,
\end{equation*}
where $a=0.08$, and $b=9.54$.
Also, $t=0$ represents the current year, 2022.
We then make the growth rate function dependent on time by replacing the temperature parameter $T$ with the temperature function, as shown bellow,
\begin{equation*}
    \begin{array}{ll}
        G(t) &= R(t) + \scalebox{1.2}{$\frac{P'(t)t}{P(t)}$} \\[.2cm]
         &= \ln\left(\scalebox{1.2}{$\frac{0.32*5}{1+10^{-4}(0.08t-2.96)^4}$}\right) - \scalebox{1.2}{$\frac{4*10^{-4}*0.08t(0.08t-2.96)^3}{1+10^{-4}(0.08t-2.96)^4}$}.
    \end{array}
\end{equation*}
Lastly, we replaced this new growth rate function with the growth rate parameter in the original logistic model, \equationautorefname~\eqref{eq:fishlogistic}, and compare its results.
We found that after approximately 100 years, the salmon population will begin to decline and eventually die off or migrate elsewhere.
In the next chapter, we will examine the effect of interaction between the brown bear and salmon species.
We will compare the results of the interaction with and without the growth rate function.