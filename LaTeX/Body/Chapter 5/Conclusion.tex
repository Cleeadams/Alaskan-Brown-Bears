In this chapter, we introduce a variation of Theodore Modis' model, \equationautorefname~\eqref{eq:modislotkavoltera}, to simulate the interaction between the brown bear and salmon species. 
We start with the autonomous system of ODEs, \equationautorefname~\eqref{eq:AutonomousSystemODEs}, and analyze its stability near its critical points.
We established that for the critical point where neither population is extinct, both eigenvalues of the Jacobian matrix contain imaginary parts and negative reals parts, which implies the fixed point is a stable spiral.
We then substituted $c_{xy}=0.0627$ and $c_{yx}=0.0313$, into the autonomous model, \equationautorefname~\eqref{eq:AutonomousSystemODEsModis}, which created an oscillation around the critical point $(0.79,\;7.1)$.
When plotting the solutions for the autonomous model, it was apparent that the brown bear species will overtake the salmon species, bringing the salmon near regional extinction.
% For the interaction parameters, $c_{xy}=0.0627$ and $c_{yx}=0.0313$, the stability around the critical point, $(0.61,7.19)$, to be asymptotically stable.
In the next section, we applied the above interaction parameters to the non-autonomous system of ODEs, \equationautorefname~\eqref{eq:Non-AutonomousSystemODEs}, which includes the growth rate function and compared its solutions to the autonomous model.
The results of the non-autonomous model were similar to the autonomous version for the first 65 years. As the temperature began to leave the optimal range, the solutions to the non-autonomous model separated from those of the autonomous version, resulting in the entire salmon population becoming regional extinct from Alaska, and the brown bear population converging to its carrying capacity.
We conclude that global warming could eventually cause the salmon population to go extinct and the brown bear population to decrease in size to accommodate for the lack of a resource, assuming the brown bear population is not affected in any other way.