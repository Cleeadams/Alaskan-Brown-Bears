% Many population growth models use variations of the Verhulst logistic growth equation to simulate the population growth of living organisms.
% James Wallace and Anastasios Tsoularis review different variations of the logistic growth equation in their article, ``Analysis of logistic growth models''~\cite{tsoularis2002analysis}.
% In this thesis, we will use variations of the logistic growth model to describe the populations of pacific salmon and Alaskan brown bears.

In this chapter, we introduce simple logistic growth models for the salmon and brown bear species.
Using information from the Alaskan Department of Fish and Game, we estimate a growth rate parameter for the salmon population.
Then, we choose the carrying capacity parameter by calculating the maximum volume of salmon for any given inshore run\footnote{Inshore runs are when salmon migrate back from the sea to spawn.} in Bristol Bay, Alaska.
We find 3 growth rates from 3 articles for the Alaskan brown bears and calculate their mean, which we use to represent their growth rate parameter~\cite{daele2010management,mclellan1989,mclellan1996}.
Also, for this model, we estimate the carrying capacity based on information from the Alaskan Department of Fish and Game (ADFG)~\cite{ADFG}.



% we begin by creating an appropriate logistic growth model for the brown bear population using growth rates from a few scholarly articles and approximating a carrying capacity from the Alaskan Department of Fish and Game (ADFG). 
% Next, Volhulst’s logistic growth model will illustrate the salmon population using information from the ADFG to achieve a growth rate. 
% Also, the carrying capacity is revealed through the relationship between salmon run size and their average weight during their run. 
% Finally, the chapter briefly compares the two models and introduces the concept of creating a variation of the logistic model that incorporates the aspect of climate change.