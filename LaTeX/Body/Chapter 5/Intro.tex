We begin this chapter by introducing the Lotka-Volterra equations, which describe the interactive relationship between two species using a system of first-order nonlinear differential equations.
The Lotka-Volterra equations, also called the predator-prey model, demonstrate that without interaction, the prey population will grow exponentially while the predator population will decay exponentially.
We then introduce Theodore Modis' competitive predator-prey model, a variation of the Lotka-Volterra equations where the logistic growth equation is the base model for both species.
In the following section, we propose a model similar to Theodore Modis' that portrays the interactive relationship between brown bears and salmon when both species are unaffected by climate change.
In the third section, we solve for the critical points of our model and determine the stability of the populations near their critical points.
We then evaluate the solutions of the model with different pairs of interaction parameters to determine the behavior of both species over time.
We fix the interaction parameters to a pair of values that create a noticeable oscillation of solutions for both species.
In the next section, we introduce climate change to our model by replacing the growth rate parameter with the growth rate function, $G(t)$.
Using different initial conditions for the model, we visualize the species' behavior over time.
Lastly, in this chapter, we compare both models, concluding that global warming will cause the salmon population in Alaska to become regionally extinct and the Alaskan brown bears to compensate for the loss of a food resource by decreasing their population size.