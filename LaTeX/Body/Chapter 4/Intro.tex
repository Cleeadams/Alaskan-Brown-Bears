In this chapter, we propose a salmon growth rate function that depends on time.
First, we use research articles to obtain river temperatures during spawning migration that indicate when salmon survival is optimal versus minimal.
Then, we use Katherine Carter's article, ``The effects of temperature on steelhead trout, coho salmon, and chinook salmon biology and function by life stage'' to extrapolate the proportion of salmon that would survive spawning migration in each of the obtained temperatures.
Next, we design a function that estimates the survival proportion of salmon during spawning migration with respect to temperature.
After that, we sample data from the United States Geological Survey (USGS) to construct a temperature growth model dependent on time.
Then, we replace the temperature parameter in the survival proportion function with the temperature model, resulting in the function being dependent on time.
Lastly, we combine this function with the current growth rate parameter, which produces the proposed salmon growth rate function.


% In this chapter, we introduce critical water temperature values and ranges essential to analyzing salmon migration patterns.
% Using these significant temperatures a function is designed to express the proportion of salmon that will survive a migration to spawning locations with respect to temperature.
% Next, the growth rate will incorporate the proportion function, thus creating a growth function dependent on temperature.
% From this point, the chapter will analyze water temperature patterns during salmon spawning season for the past 30 years to fit a linear model to the change in average water temperature over time.
% Finally, the chapter compares the new model for the salmon population to the models discussed thus far.
