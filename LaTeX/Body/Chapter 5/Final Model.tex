Currently, we have constructed models for both species that represent their populations without the influence of each other. 
% The models found in \equationautorefname~\eqref{eq:LogBear} and \equationautorefname~\eqref{eq:salmonlogisticrepo} are designed to individually represent the species' populations.
% This means that the outcome of one species does not affect the other.
By including interaction terms for both models, we can simulate a trade-off of environmental resources, as we would see in the real world.
First, we use a variation of Theodore Modis' model, \equationautorefname~\eqref{eq:modislotkavoltera}, to introduce interaction between the brown bears and salmon when both species are unaffected by climate change,
\begin{equation}\label{eq:AutonomousSystemODEs}
    \begin{aligned}
    \frac{dx}{dt} &=r_xx\left(1-\frac{x}{K_x}\right) - c_{xy}xy,\\[.4cm]
    \frac{dy}{dt} &=r_yy\left(1-\frac{y}{K_y}\right) + c_{yx}xy,
    \end{aligned}
\end{equation}
where $r_x = \ln(0.32*5),\ r_y=0.059,$ and $c_{xy},\;c_{yx}>0$. \equationautorefname s~\eqref{eq:fishlogistic} and~\eqref{eq:LogBear} represent the base models of the above system of differential equations where both species are unaffected by climate change.
Notice, for the salmon ODE, we subtract its interaction term, but for the brown bears, we add its interaction term.
We do this because the salmon population should have a negative consequence when there is an increase in brown bears.
In contrast, brown bears should be rewarded when their food source increases.
The interaction parameters alter the effect of the carrying capacity, so we change $K_x = 15$ and $K_y=5$ as a vague measure of their environmental limits.
We can rewrite the system of equations in a similar form to Theodore Modis' model, as shown below,
\begin{equation}
    \begin{aligned}
    \frac{dx}{dt} &= r_xx -\frac{r_xx^2}{K_x} - c_{xy}xy,\\[.4cm]
    \frac{dy}{dt} &=r_yy -\frac{r_yy^2}{K_y} + c_{yx}xy.
    \end{aligned}
\end{equation}
Then, substituting $ a_x = r_x$, $\displaystyle b_x = \frac{r_x}{K_x}$, $a_y = r_y$, and $\displaystyle b_y = \frac{r_y}{K_y}$ we get
\begin{equation}\label{eq:AutonomousSystemODEsModis}
    \begin{aligned}
    \frac{dx}{dt} &= a_xx - b_xx^2 - c_{xy}xy,\\
    \frac{dy}{dt} &= a_yy - b_yy^2 + c_{yx}xy.
    \end{aligned}
\end{equation}
Condensing the model reduces the number of parameters in each equation, which makes the model and any operations done to the model more readable and interpretable.
In the next section, we determine how different values for the parameters, $c_{xy}$ and $c_{yx}$, control the stability of the populations near their critical points.
