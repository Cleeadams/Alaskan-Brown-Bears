% The topic of climate change has been debated for many years. 
% Some people believe that global warming is a myth and misuse scientific research to support their claims~\cite{allchin2015global}. 
% % Thomas Crowley constructed a model using recorded weather data to determine temperature values up to 1000 years ago~\cite{crowley2000causes}.
% Many researchers have concluded that the earth's temperature has increased significantly since the early 20$^{th}$ century and will continue to increase until at least the mid to late 21$^{st}$ century~\cite{hansen2006global,raftery2017less,osterkamp1990thermal}.
% A group of researchers implemented a joint Bayesian hierarchical model, which determined that the global temperature will likely increase by an average of $3.2^{\circ}$C entering the 22$^{nd}$ century~\cite{raftery2017less}.
% This significant growth in temperature is due to the increase in greenhouse gases, such as carbon dioxide, methane, nitrous oxide, and fluorinated gases \cite{crowley2000causes, osterkamp1990thermal, epagreen}.
% Greenhouse gases fill the air and act as a canopy that prevents heat from escaping the atmosphere, causing an escalation in climate temperature~\cite{epagreen}. 
% In 2019, the EPA reported that 80\% of the earth's greenhouse gases are carbon dioxide~\cite{epagreen}.
% Because of the high demand for transportation, electricity, industrial production, and residential/commercial use of fossil fuels, carbon dioxide will continue to be the primary source of greenhouse gas emissions~\cite{epagreen}.
% Trees, plants, and the ocean are part of the carbon cycle, which removes massive amounts of carbon dioxide from the atmosphere.
% Carbon dioxide emissions have been declining slowly over the past 15 years, so global temperature trends may decrease in the near future~\cite{epagreen}.

% % The effects that will follow this dramatic temperature change will raise the average sea level, cause animals to migrate to new locations, threaten the population of animals that are sensitive to their climate conditions ~\cite{osterkamp1990thermal,adams2011modelling,taylor2008climate}

% % Global warming becomes more serious as the years keep rolling by. The most recent wake up calls are the effects on polar and koala bears \cite{wiig2008effects, adams2011modelling, stirling2012effects}. The permafrost in Alaska is melting which is causing damage to people's homes, roads, and pipelines
% % \cite{osterkamp1990thermal}. The population of polar bears are in a decline due to the melting of ice caps \cite{hunter2010climate}.
% % The koala bears in Australia had to migrate to new locations due to the increase in wildfires~\cite{adams2011modelling}.
% % These are just a few significant events that are breaking out in the world due to global warming, but we can expect to see more as time goes on.
% % Animals that are sensitive to their environment, such as many sea life, are under threat because the sea surface temperature is changing alongside the air temperature \cite{taylor2008climate,hansen2010global,iz2018global}.
% % For example, Salmon in Alaska are experiencing higher rates of prespawning mortality because they are migrating to the creeks earlier due to the shift in their environmental temperatures \cite{bowerman2018prespawn}. 
% % If the temperatures continue to increase, then prespawning mortality will increase as well \cite{bowerman2018prespawn}.
% % Scientists around the world have researched the causes for the sudden climb in temperature in hopes to find a solution to reverse the effects \cite{epagreen,cook2016consensus}.
