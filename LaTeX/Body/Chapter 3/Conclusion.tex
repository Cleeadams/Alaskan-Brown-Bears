In this chapter, we found the growth rate and carrying capacity for the salmon and brown bear species to be: $r_x=\ln(0.32*5)\approx0.47$, $K_x=29.1*10^6$, $r_y=0.059$, and $K_y=4.5*10^{4}$.
A logistic model was used to represent both species, as shown below,
\begin{equation*}
    \frac{dx}{dt} = r_xx\left(1-\frac{x}{K_x}\right),
\end{equation*}
and
\begin{equation*}
    \frac{dy}{dt} = r_yy\left(1-\frac{y}{K_y}\right).
\end{equation*}
The solutions to these models show that the salmon species reaches its environmental limit in 14 years and the brown bears reach theirs in 100 years.
Now that each species has a foundational model for their growth behavior, in the next chapter, we will create a function dependent on time that replaces the growth rate parameter for salmon.