Climate change has been a popular topic since James Hansen gave his testimony to Congress in 1988, expressing the disasters that would come from global warming.
Many researchers are studying climate change in hopes of predicting its effects.
If we can anticipate the outcomes of climate change, we can take measures to minimize or eliminate the catastrophes that will follow.
In this thesis, we compare two models that determine the long-term outcome of two interactive species, pacific salmon \emph{Oncorhynchus} and Alaskan brown bears \emph{Ursus arctos}.
The first model predicts the outcome of the species when temperature is constant, and the other when temperature is a function of time.
We conclude that the effects of global warming could cause the pacific salmon to either die off or migrate to an area that is more suitable for their environmental needs, resulting in the brown bear population decreasing in size to accommodate for the elimination of a food source.