In the analysis of this model, we will solve for the equilibrium points and report the outcomes at those points as well as what happens in between.

First, we can find the critical points of the model by solving the system of nonlinear equations when the derivative is equal to zero.
\[
\begin{aligned}
    0 &=R(T)F\left(1-\frac{F}{K_F}\right) - c_{FN}FN\\[.4cm]
    0 &=rN\left(1-\frac{N}{K_N}\right) + c_{NF}FN
\end{aligned}
\]
The most obvious is points are $(0,0)$, $(K_F,0)$, and $(0,K_N)$.
The next equilibrium point occurs is when:
\[
-c_{NF} FN= rN\left(1-\frac{N}{K_N}\right)
\]
\[
-c_{NF}F = r\left(1-\frac{N}{K_N}\right)
\]
\[
F = \frac{r}{-c_{NF}}\left(1-\frac{N}{K_N}\right)
\]
By applying the same steps above to the top equation in the model we get:
\[
N=\frac{R(T)}{c_{FN}}\left(1-\frac{F}{K_F}\right)
\]
Now, treating these as a system of linear equations will make finding the equilibrium much easier.
\[
\begin{aligned}
    F - \frac{rN}{c_{NF}K_N} &= -\frac{r}{c_{NF}}\\[.4cm]
    \frac{R(T)}{c_{FN}K_F}F + N &= \frac{R(T)}{c_{FN}}
\end{aligned}
\]
Multiply the bottom row by $\frac{c_{NF}}{r}$ and add to the top, we get:
\[
F\left(1+\frac{rR(T)}{c_{NF}K_Nc_{FN}K_F}\right) = \frac{rR(T)}{c_{NF}K_Nc_{FN}}-\frac{r}{c_{NF}}
\]
Now, solving for F we get:
\[
F = \frac{rc_{NF}\left(R(T)-K_Nc_{FN}\right)}{c_{NF}^2K_Nc_{FN}} * \frac{K_NK_Fc_{NF}c_{FN}}{K_NK_Fc_{NF}c_{FN}+rR(T)}
\]
\[
F = \frac{rK_F(R(T)-K_Nc_{FN})}{K_NK_Fc_{NF}c_{FN}+rR(T)}
\]
Then, solving for $N$, we get:
\[
N = \frac{R(T)}{c_{FN}}\left(1-\frac{\frac{rK_F(R(T)-K_Nc_{FN})}{K_NK_Fc_{NF}c_{FN}+rR(T)}}{K_F}\right)
\]
\[
N = \frac{R(T)}{c_{FN}}\left(1-\frac{K_F(K_NK_Fc_{NF}c_{FN}+rR(T))}{rK_F(R(T)-K_Nc_{FN})}\right)
\]
\[
N = \frac{R(T)}{c_{FN}}\left(\frac{rK_F(R(T)-K_Nc_{FN})-K_F(K_NK_Fc_{NF}c_{FN}+rR(T))}{rK_F(R(T)-K_Nc_{FN})}\right)
\]
\[
N = \frac{R(T)}{c_{FN}}\left(\frac{-rK_Nc_{FN}-K_NK_Fc_{NF}c_{FN}}{r(R(T)-K_Nc_{FN})}\right)
\]
\[
N = \frac{R(T)K_N(-r-K_Fc_{NF})}{r(R(T)-K_Nc_{FN})}
\]
\[
N = \frac{R(T)K_N(r+K_Fc_{NF})}{r(K_Nc_{FN}-R(T))}
\]
From here, we can establish that there is an critical point at:
\[
\left(\frac{rK_F(R(T)-K_Nc_{FN})}{K_NK_Fc_{NF}c_{FN}+rR(T)}, \frac{R(T)K_N(r+K_Fc_{NF})}{r(K_Nc_{FN}-R(T))}\right)
\]
The beauty of this point is that it contains the reproduction function, $R(T)$, which implies that the point moves as temperature, $T(t)$, changes, which will as time, $t$, increases.

Now, to evaluate what happens at these critical points, we will be looking at the Jacobian matrix of the model.
First, let:
\[
\begin{aligned}
    f_1(F,N) &= R(T)F\left(1-\frac{F}{K_F}\right) - c_{FN}FN\\[.4cm]
    f_2(F,N) &= rN\left(1-\frac{N}{K_N}\right) + c_{NF}FN
\end{aligned}0
\]
So, the Jacobian is below:
\[
\mathbf{J}_{(F,N)} =
\begin{bmatrix}
  \frac{\partial f_1}{\partial F} & 
    \frac{\partial f_1 }{\partial N}\\
  \frac{\partial f_2}{\partial F} & 
    \frac{\partial f_2}{\partial N}
\end{bmatrix}
=
\begin{bmatrix}
  R(T)-2\frac{R(T)F}{K_F}-c_{FN}N & 
    -c_{FN}F\\
  c_{NF}N & 
    r-2\frac{rN}{K_N}+c_{NF}F
\end{bmatrix}
\]
Now, the Jacobian matrix for the critical points are:
\begin{enumerate}
    \item $(0,0)$
    \[
    \mathbf{J}_{(0,0)} =
    \begin{bmatrix}
  R(T) & & &
    0\\
  0 & & & r
    \end{bmatrix}
    \]
    \item $(K_F,0)$
    \[
    \mathbf{J}_{(K_F,0)} =
    \begin{bmatrix}
  -R(T) & & & -c_{FN}K_F\\
  0 & & & r+c_{NF}K_F
    \end{bmatrix}
    \]
    \item $(0,K_N)$
    \[
    \mathbf{J}_{(0,K_N)} =
    \begin{bmatrix}
  R(T)-c_{FN}K_N & & & 0\\
  c_{NF}K_N & & & -r
    \end{bmatrix}
    \]
    \item $(F_c, N_c)=$ $\left(\frac{rK_F(R(T)-K_Nc_{FN})}{K_NK_Fc_{NF}c_{FN}+rR(T)}, \frac{R(T)K_N(r+K_Fc_{NF})}{r(K_Nc_{FN}-R(T))}\right)$
    \[
    \mathbf{J}_{(F_c,N_c)} =
    \begin{bmatrix}
  R(T)-2\frac{R(T)F_c}{K_F}-c_{FN}N_c & & & 
    -c_{FN}F_c\\
  c_{NF}N_c & & &
    r-2\frac{rN_c}{K_N}+c_{NF}F_c
    \end{bmatrix}
    \]
\end{enumerate}

To analyze what happens at each critical point, we can look at the eigenvalues of the Jacobian matrix. Notice with the first critical point, $(0,0)$, the Jacobian matrix is diagonal which implies that the eigenvalues are the diagonal values:
\[
\lambda_1=R(T),\quad \lambda_2=r
\]
If 
The next critical point, $(K_F,0)$
