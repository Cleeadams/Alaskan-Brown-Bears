In this thesis, the Alaskan brown bear and pacific salmon populations were modeled using a combination of the logistic growth equation, Lotka-Volterra equations, and a tailored function that simulates the salmon growth rate with respect to time.
The models we constructed were used to compare outcomes of the species' populations when  climate temperature is stable versus increasing with time.
We found that both species were affected by global warming, with the salmon population facing dangers of extinction.
Our models are designed to reflect the behaviors of each species, where the growth rate function plays a vital role in controlling the outcomes of both populations.
In the future, we plan to explore different growth rate functions that may provide a more accurate representation of their behaviors.
A growth rate function we considered was:
\begin{equation*}
    G(t) = \ln\left(\frac{5}{1+1^{-4}(0.08t-2.96)^4}\right) - 0.68,
\end{equation*}
where we subtract the mortality rate of the salmon harvested by commercial fisheries instead of multiplying the reproduction rate of salmon by its survival rate~\cite{NPS}.
While this is one other way of interpreting the growth rate, there are a many more to explore and incorporate into our model.

